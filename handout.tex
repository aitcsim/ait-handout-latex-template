%%%%%%%%%%%%%%%%%%%%%%%%%%%%%%%%%%%%%%%%%%%%%%%%
% 
% == AIT CSIM Handout LaTeX Template ==
% == Credit ==
% Assoc. Prof. Matthew N. Dailey
% Computer Science and Information Management
% Asian Insitute of Technology
% 
%%%%%%%%%%%%%%%%%%%%%%%%%%%%%%%%%%%%%%%%%%%%%%%%

\documentclass{article}

\usepackage{a4,url,upquote}
\usepackage{graphicx}
\usepackage{hyperref}
\usepackage[cmex10]{amsmath}
\usepackage{amssymb}
\usepackage{placeins}

\setlength{\textwidth}{6.5in}
\setlength{\textheight}{9in}
\setlength{\oddsidemargin}{0in}
\setlength{\evensidemargin}{0in}
\setlength{\topmargin}{0in}
\setlength{\headheight}{0in}
\setlength{\headsep}{0in}
\setlength{\footskip}{0.5in}

\newcommand{\bheading}[1]{\vspace{10pt} \noindent \textbf{#1}}

\begin{document}

\begin{tabbing}
    \`\=\kill
    \textbf{Workshop:} Workshop Name
    \` September 11, 2015 \\
    Asian Institute of Technology
    \` Computer Science and Information Management \\
    \textbf{Handout:} Workshop Title
    \` \textbf{Instructor:} Matthew N.\ Dailey (\tt{\small mdailey@ait.asia})
\end{tabbing}

\hrule

\vspace{.25in}

\begin{center}
    \textbf{\Large Workshop Title}
\end{center}

\vspace{.15in}

\noindent \textbf{Instructions:} In this workshop, you will learn how to do something.

% Inserting a tilde into LaTeX
% Credit:
% http://tex.stackexchange.com/questions/9363/how-does-one-insert-a-backslash-or-a-tilde-into-latex

\subsection*{Section 1}

\begin{itemize}
    \item[-] {\tt code} - Folder that contains code and scripts
    \begin{itemize}
        \item[-] File {\tt workshop1.py}
        \item[-] File {\tt workshop2.py}
    \end{itemize}
    \item[-] {\tt data} -- Folder that contains the datasets.
    \begin{itemize}
        \item[-] {\tt input} -- Folder that contains the input images.
        \begin{itemize}
            \item[-] {\tt folder\_1} -- Folder that contains some files.
            \item[-] {\tt folder\_2} -- Folder that contains some other files.
        \end{itemize}
    \end{itemize}
\end{itemize}

\FloatBarrier

\subsection*{Section 2}

\noindent We will do a simple workshop. Figure~\ref{fig:csim-logo} shows
a logo of CSIM. \\

\begin{figure}[t]
    \centering
    \includegraphics[width=2in]{figures/csim}
    \caption{CSIM Logo}
    \label{fig:csim-logo}
\end{figure}

\noindent Let's follow these steps below.

\begin{enumerate}
    \item Step 1
    \item Step 2
    \item Step 3
\end{enumerate}

\noindent The logo of CSIM is from the CSIM Web site~[1]. \\

\subsection*{References}

\begin{itemize}
  \item[1] \tt{http://www.cs.ait.ac.th/}
\end{itemize}

\end{document}
